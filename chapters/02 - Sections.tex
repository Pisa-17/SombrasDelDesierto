\section{El eco de los antiguos}
\subsection{Mercado de Karnak - Desierto de Amon-Ra}
\paragraph{\textit{En las arenas doradas del desierto de Amon-Ra, donde el sol quema sin piedad, y las noches traen promesas de misterio, un joven mercenario llamado Setekh buscaba su próximo bocado
No era un ladrón por elección, sino por necesidad, y en las sombras del mercado de Karnak, su destino se estaba tejiendo en silencio}\\}

El mercado, aquel día se encontraba realmente abarrotado de gente, casi no cabía un alfiler en tan pequeñas calles de la ciudad, y Setekh lo sabía perfectamente. Hoy era uno de esos días en los que,
él podía ganar bastante dinero y lo más importante, un bocado al final del día, era consciente de que hoy, teniendo tales facilidades, los mercaderes, estaría alerta de personas como él. Pero, como dijo el mismo:\\
\textit{- Quien no arriesga, no gana, y hoy debo ir con todo.}\\
Y dentro de esa multitud de gente alrededor de los preciados productos de los mercaderes, se encontraba él, fluyendo entre la gente, y poco a poco haciéndose una pequeña fortuna mientras seguía hacia delante,
a paso lento, pero tampoco muy rápido, simplemente se estaba dejando llevar, entre tanta gente, y estaba llegando al final de la avenida, pocos pasos le separaban de acabar de amasar una fortuna y poder ir libremente,
a poder tener su preciado bocado. Pero una voz detuvo ese paso fluido como si de un baile se tratara, alguien apagó la música en tal \"bella arte\" que tenía el mercenario. \\
\textit{-¡Alto ahí! Suelta todo lo que llevas, te hemos visto desde hace un buen rato, pagarás tus malos actos frente a la justicia.}\\
Setekh, se paró y como si del mismo diablo se tratase, su cara, quedó fría, en aquel ambiente sofocante, donde hasta el mismo aire pesaba del calor que hacía, mientras escuchó aquellas palabras, vislumbró dos guardias dirigiéndose hacia él, y sabía perfectamente
que en su espalda habría otros dos yendo a por él. Se encontraba en una encrucijada de la que era dificil salir, por lo que Setekh, pensó rápido y corrió hacia un callejón, siendo un mercenario y ladrón, debe saberse cada hueco de la ciudad, pero los guardias, igualmente,
conocen cada rincón, y siguieron al mercenario. Sus ropajes de color azabache claro, delataban su presencia, y su mirada entre su turbante, anunciaba sus intenciones, ahora mismo, el juego había cambiado, Setekh debía escapar y esconderse, mientras los guardias seguían tras él,
pero como suelen decir, la codicia, es pesada y Setekh, lo estaba experimentando, debía dejar parte de las monedas de oro que había robado, o de lo contrario, los guardias lo atraparían tarde o temprano, así que, eso hizo, dentro de un ladrillo de la ciudad, escondió una bolsa
con varias monedas de oro, que más tarde recogería, así ahora Setekh podría correr sin preocuparse por ir más lento, pero, el juego del gato y el ratón estaba cerca de su final, y nuestro mercenario, estaba perdiendo la ronda. Setekh estaba siendo más rápido que los guardias, pero
estos, erán muy astutos y lograban bloquear cada callejón al que Setekh iba a ir para escapar, el mismo mercenario no tardó en darse cuenta que poco a poco, estaban cercando su posición y pelear en sus condiciones con un guardia, solo le traería la ruina, solamente quedaba una opción
y era entregarse, pero si lo hacía, sería probable que durmiera en las catacumbas de la ciudad. La situación era realmente complicada y Setekh poco a poco, gracias a su velocidad, iba esquivando los guardias, pero, ¿cuánto tiempo podría estar esquivando mientras el cerco se hacía más pequeño?,
el joven mercenaria no veía salida, cada vez, estaba más rodeado y los pequeños callejones disminuían conforme la guardia iba tras él, estaba cerca de ser capturado, pero, una voz que escuchó, le tendió la mano:\\
\textit{-Parece que vas con prisa, por aquí no vendrán.} \\
Una joven, se dirigió al mercenario para ofrecerle entrar a una pequeña casa entre callejones, a lo que el mercenario, dubitativo no respondió, pero la joven sí:\\
\textit{-No me lo pensaría demasiado, no parece que tengas tiempo, si quieres seguir sin pisar las catacumbas, sígueme}\\
El mercenario, siguió la joven que parecía ser incluso más rápida que él, y usaba un extraño paso, que le fascinó al mercenario, aquellos pasos eran veloces como el viento, y suaves como la brisa de primavera, el mercenario podía mantenerse a su ritmo y pronto esquivando la guardia, llegaron
al lugar donde la joven quería llegar. Una pequeña casa austera, y tranquila, unas calles más abajo de donde se encontró con el mercenario, una vez dentro, la joven se quitó su ropaje de tela que le cubría el rostro y parte del que le cubría el pelo, tras esas prendas de seda, se escondía un largo cabello plateado,
que a la luz del sol brillaba con destellos, junto con unos ojos color verde esmeralda, tan claros como el agua de un río, la joven, mientras dejaba ver su rostro advirtió al mercenario:\\
\textit{-He visto que tienes las manos muy ligeras, caballero, pero dejame decirte que tengo algo que has dejado por ahí tirado, francamente, se te da mal esconder cosas}\\
Tras esto, la joven dejó caer la bolsa de monedas robadas del mercenario, sobre el suelo, a lo que el mercenario quedó sorprendido:\\
\textit{-¿En qué momento has sido capaz de recogerlas?}\\
La joven, con una sonrisa confiada y un brillo pícaro en sus ojos, respondió:\\
\textit{-Vaya, menudos modales. Lo primero que te oigo decir y ni siquiera te presentas.} \\
El mercenario, desconocía la identidad de la mujer, pero aun así agradeció su gesto:\\
\textit{-Tienes razón, además estoy en deuda contigo, soy Setekh, mercenario del desierto norte.}\\
La mujer, sonrió y se dió la vuelta para rebuscar algunas cosas mientras hablaba con el mercenario:\\
\textit{-No hace falta que seas tan correcto, solamente bromeaba contigo, te he estado observando un tiempo en el mercado y preveía que tendrías problemas, además, hay alguien que tiene interés en ti, por lo que,
tarde o temprano debería conocerte en persona, \"Viento desértico del norte\"}\\
Los ojos del mercenario, en un momento se volvieron platos, su sorpresa fue mayúscula pero mantuvo la compostura:\\
\textit{-¿Cómo conoces ese nombre?, ¿Quién eres?, ¿Quién me busca?}\\
La mujer volvió a mirarle y le dijo:\\
\textit{-Para ser tan misterioso, preguntas bastante, pero, ¿volvemos al interrogatorio?, eres curioso}\\
Setekh, no parecía entender muy bien que es lo que deseaba esta chica, y presentía algo malo en todo esto, su mente podía estar jugandosela, pero quizá no era una mala idea confiar en la desconocida mujer:\\
\textit{-Puedes prescindir de tus ropajes, no hace falta que ocultes más tu rostro, aquí dentro no hay peligro, soy Nefertari, pertenezco a una organización que busca personas con talentos ocultos. Nuestro líder cree que tienes un potencial especial, uno que ni siquiera tú conoces todavía. Mi tarea era encontrarte y traerte con nosotros.}\\
Setekh frunció el ceño, evaluando las palabras de Nefertari. La desconfianza se mezclaba con la curiosidad en su mirada:
\textit{-¿Y por qué debería confiar en ti? Podría ser una trampa}\\
\textit{-No te pido que confíes ciegamente. Te ofrezco una oportunidad para descubrir quién eres realmente y para qué estás destinado. Si aceptas, podrías encontrar respuestas y poder que nunca imaginaste. Si no, eres libre de irte con tus monedas robadas.}\\
Setekh se quedó en silencio por un momento, pensando en su vida hasta ahora y en las posibilidades que Nefertari ofrecía. Había dos opciones sobre la mesa y el joven mercenario debía decidir, quizás una decisión que encaminaría su vida a algo más grande. La primera opción parecía bastante intersante pero Setekh aun desconfía de una desconocida, aunque la segunda opción no era mejor, por lo que Setekh suspiró y con mucha calma dijo sus condiciones:\\
\textit{-De acuerdo. Te seguiré. Pero si esto es una trampa, te aseguro que no será fácil atraparme}\\
\textit{-No esperaba menos de ti, Setekh. Ven, tenemos mucho que discutir y preparar.}\\

\clearpage 
% ------------------------------------------------------
Un tiempo más tarde...
Nefertari y Setekh, se conocieron algo más y ella, le otorgó más detalles acerca de la organización de la que formaba parte, conforme pasaba el tiempo, Setekh estaba más convencido de entrar a la organización y Nefertari, dejaba ver muestras de felicidad mientras Setekh se mostraba más convencido. Ambos esperaron a la noche y Nefertari, le dijo a Setekh que deberían partir por el paso subterráneo de la ciudad, a lo que Setekh replicó, alegando que siempre hay guardias, lo que no sabía Setekh,
es que Nefertari se refería a otro paso subterráneo, uno, más escondido:
\textit{-Por aquí, sígueme}\\
Setekh seguía a Nefertari, sin ninguna objección y estaban yendo en dirección oeste de la ciudad, el paso subterráneo, se encontraba en la parte norte, pero Setekh se limitó a seguir a Nefertari. Pronto llegaron a un pequeño monumento en honor a Khonsu, el dios egipcio de la luna y la noche. El monumento era una pequeña fuente con una piedra que por las noches, emite un brillo especial que baña el agua de la misma en un color verde bosque por su reacción con la luz de la luna. Nefertari dejó caer en el suelo un puñado de arena que sacó de una bolsa, y puso su mano boca abajo encima del lugar donde cayó la arena, se arrodilló y pronto se iluminó poco a poco un circulo alrededor de ella y Setekh, era un circulo color cobre con algunos jerogrificos
escritos, poco a poco se fue levantando la arena que Nefertari había soltado, y fue rodeando a los dos, Nefertari se levantó y dijo:\\
\textit{-Ahora se abrirá una puerta, deberemos pasar rápido por ella, simplemente no dudes, necesito que pases con convicción, de lo contrario, no llegaremos al paso}\\
Setekh asintió, y tan pronto como lo hizo, el suelo de sus pies, empezó a ceder y bajar. Setekh se sorprendió pero Nefertari le pidió que no se moviera, estaban entrando al paso subterráneo secreto del oeste. El suelo siguió bajando hasta que llego la plataforma del paso, donde se encontraban dos antorchas iluminando la entrada, Nefertari, salió de la plataforma y Setekh la siguió, ahí se encontraba otra puerta más, ante la cual, Nefertari y le dijo:\\
\textit{- ¿Sabes?, sé que es repentino y quizás no lo esperas, pero, a partir de aquí, ten cuidado, iremos hasta la \"Luna Creciente\", aunque te voy a acompañar, cruzar este paso, no es solo un camino que debemos recorrer, es una decisión de no volver atrás, por eso quiero, que pase lo que pase, te mantengas fuerte}\\
Setekh solamente asintió y cruzó la puerta con ella, dentro del paso, la presión era inmensa, apenas se podía respirar debido a una presión de poder que recorría todo el paso, Nefertari se estremeció un poco y dijo:\\
\textit{- Lo que sientes es presión de poder pura, viene del centro de la gran pirámide, por aquí pasa una raíz de poder de ella, de lo único de lo que nos debemos preocupar es de la acumulación de armadura de corrupción que hay por este lugar, simplemente mantente alejado, y no correrás peligro}\\
Setekh y Nefertari, siguieron el camino del paso, sin mucha prisa, a un paso calmado, ya que la presión de poder apenas permitía ir más rápido, aun así, llegaron al fin del paso, en el que esperaba, de nuevo una puerta, pero esta, era distinta, requería usar una llave.
Nefertari sacó de su bolsa, una llave con la forma de una luna creciente, que introdujo en la cerradura de la puerta lentamente y abrió sin mucho esfuerzo, ambos pasaron la puerta y Nefertari, se aseguró de cerrar con llave aquella puerta, la presión de poder seguía ahí pero no era tan grande como en el paso.\\
\textit{- Siempre es complicado pasar por aquí por culpa de la presión, a veces la gente se ha desmayado ya que no son capaces de aguantar la presión}\\
Es entendible, ya que la presión de una de las raíces es tan grande que puede dejar sin conocimiento la gente más débil. Incluso los integrantes de la \"Guardia del Sol\" temida por sus formidables guerreros, no se les permite dejar libremente su poder cuando hacen acto de presencia, ya que su presión puede llegar a ser similar a la de una sola raíz, cuentan las leyendas que existe una raíz por cada guerrero especial de la \"Guardia del Sol\". Algunos eligen creer esas historias y otros dudan de ellas.\\
Al llegar a la salida del pasaje subterráneo, Setekh notó que estaba en las afueras de la ciudad, cerca de la entrada al desierto oeste de la zona. Este desierto es conocido, o más bien desconocido, por todos. Pocas son las personas que han llegado a él y han podido volver. Las tormentas de arena en esa parte del desierto son comunes, tanto que se piensa que una fuerza de las pirámides actúa sobre esa parte y solo permite a los más poderosos entrar y salir.\\
Nefertari miró a Setekh y asintió, indicándole que entrarían, pero no por el lugar habitual.\\
\textit{- Vamos, Setekh. Entraremos por una entrada diferente. Sigue mis pasos y mantente alerta}
El desierto de Amon-Ra es vasto y dividido en varias zonas. La zona oeste es la más enigmática, ya que es complicada de explorar. En cuanto al resto, la zona norte habitualmente alberga a la población pobre, donde suelen haber bastantes problemas debido a la falta de recursos. En el este, se encuentra la frontera general con las Tierras de Hielo de Frostheim. Aunque para llegar a esas zonas gélidas es necesario seguir un largo camino, la frontera con Amon-Ra se encuentra al final del desierto del este. Por último, en la parte sur, se encuentra el puerto, una zona próspera económicamente que recibe productos de todo el mundo y permite el paso de muchos mercaderes. Existe un camino comercial desde el puerto a la ciudad central de Ankhara, el lugar donde Setekh conoció a Nefertari. La ciudad de Ankhara, es una ciudad muy rica en cultura, sociedad y dinero, esconde algún secreto a primera vista pero es gobernada por el faraón Amon IV, desciente del fundador del gran desierto Amon-Ra.\\
En el desierto oeste, existen varias entradas debido a su enigmática naturaleza. Estas están numeradas del 1 al 7, siendo la entrada 1 la más segura y la 7 la menos. Antiguamente, existían más entradas, pero se deshicieron todas las restantes menos una, la entrada maldita del desierto del oeste. Incluso la propia guardia de la ciudad apenas se atreve a merodear cerca de la misma.\\
\textit{- ¿Vamos a entrar por la entrada maldita?}\\
Nefertari sonrió ligeramente y dijo:
\textit{- Creo que aun no tienes la suficiente soltura para seguir por ese camino. Vamos a la entrada 3. Es peligrosa, pero manejable si sabes cómo moverte}\\
Avanzaron con cuidado, el viento comenzando a levantarse a su alrededor mientras se acercaban al borde del desierto. La arena crujía bajo sus pies y el aire se volvía más denso con cada paso que daban. Sin embargo, al llegar a la entrada 3, vieron a la Guardia del Sol, un grupo de soldados de élite del faraón, patrullando la entrada. Nefertari frunció el ceño y miró a Setekh con preocupación.\\
\textit{- Esto no es bueno. La Guardia del Sol no suele patrullar esta zona a menos que estén buscando algo o a alguien}\\
\textit{- ¿Qué hacemos ahora?} Preguntó Setekh.\\
Nefertari observó a su alrededor, evaluando rápidamente sus opciones. Finalmente, tomó una decisión.\\
\textit{- No tenemos otra opción. Iremos por la entrada maldita. Las otras entradas nos proporcionan caminos demasiado largos. Sígueme y no te separes.} \\
Con determinación, Nefertari guió a Setekh hacia la entrada 9, conocida como la entrada maldita. Mientras se acercaban, la atmósfera se volvía más pesada y opresiva. El viento rugía con fuerza, levantando nubes de arena que dificultaban la visión.\\
\textit{- Mantente cerca de mí. Esta entrada es peligrosa, pero confío en que podemos atravesarla si nos mantenemos juntos.}\\
Setekh asintió, sintiendo una mezcla de temor y emoción. A medida que se adentraban en la entrada maldita, las sombras parecían alargarse y cobrar vida. La sensación de peligro era palpable, pero Setekh se mantuvo firme, siguiendo a Nefertari sin dudar. Las tormentas de arena arremolinaban alrededor de ellos, dificultando el avance. Nefertari se movía con agilidad y seguridad, su conocimiento del terreno y su habilidad para leer las corrientes de viento les permitía esquivar los peores embates de la tormenta. Sin embargo, cada paso se volvía más difícil, y el camino más incierto.

Finalmente, después de lo que parecieron horas de lucha contra la arena y el viento, llegaron a un lugar más tranquilo, un oasis oculto entre las dunas. Nefertari se detuvo y señaló un conjunto de antiguas ruinas parcialmente enterradas.
\textit{- Hemos llegado. Estas ruinas nos llevarán al refugio. Pero debemos ser rápidos, las tormentas no tardarán en alcanzarnos de nuevo, y puede que ya nos sigan.} \\
Llegaron a la entrada de las imponentes ruinas. Solamente su presencia imponía respeto, con sus enormes columnas talladas con antiguos jeroglíficos y estatuas medio enterradas que contaban historias de un pasado olvidado. Nefertari, ya familiarizada con el lugar, no se detuvo a admirar los detalles. Con una urgencia palpable en su voz, le pidió a Setekh que se diera prisa.\\
\textit{- No tenemos tiempo que perder, Setekh. Debemos entrar antes de que las tormentas nos alcancen.}\\
Setekh asintió y siguió a Nefertari hasta la entrada, pero el ambiente al llegar era extraño. Las sombras abundaban en el lugar y parecían más grandes de lo normal, alargándose y retorciéndose de manera antinatural. La temperatura descendió súbitamente, y un escalofrío recorrió su espalda. Nefertari se dio cuenta de inmediato del peligro\\
\textit{- Las sombras del desierto nos han alcanzado. Debemos movernos rápido y con cuidado.} - con una voz tensa.\\
Las sombras del desierto, según las leyendas, eran entidades oscuras que se alimentaban de la desesperación y el miedo de los viajeros. Eran capaces de seguir a aquellos que se adentraban en el desierto sin el respeto adecuado, y muchos creían que eran guardianes de secretos olvidados y tesoros escondidos.\\
\textit{- ¿Qué hacemos ahora? ¿Cómo nos deshacemos de ellas?, Desde que el discipulo de dios de la muerte perdió su vida, sus sombras campan por el desierto oeste}\\
Nefertari algo tensa, respondió al joven mercenario:\\
\textit{- No podemos arriesgarnos a ignorarlas directamente. Mantente cerca de mí y no te separes, vamos a combatir.}\\
Nefertari sacó su arma, una daga brillante como la luna y muy afilada. Setekh contaba también con una daga, pero algo más rudimentaria. Nefertari, percatándose de esto, le lanzó a Setekh una nueva daga, más afilada y con una hoja de color negro obsidiana.
\textit{- ¡No dejes que se adieran a tu espiritu, eso te provocaría problemas!} - gritó Nefertari mientras iba a una velocidad impresionante.\\
Setekh se puso en guardia, hizo algún salto de calentamiento y se preparó para empezar a atacar, Setekh era muy hábil en combates debido a su elevada velocidad, pero Nefertari la superaba con creces, aun así Setekh probó a realizar varias fintas por el centro mientras que Nefertari se ocupó de atacar desde el exterior, poco a poco, fuero tumbando sombras,
hasta que de repente las pocas que quedaban, volvieron al suelo.
De repente, apareció una sombra que era una réplica exacta de Setekh. Tenía la misma altura, misma ropa e incluso la misma arma. Nefertari se dio cuenta de esto y lanzó un ataque hacia esta sombra, sorprendiendo a Setekh con la velocidad que alcanzó. Fue un ataque realmente rápido que la sombra detuvo sin mucho problema.

Setekh, sorprendido, intentó atacar a la sombra para liberar a Nefertari. La sombra, al ver que Setekh la atacaba, empezó a esquivarlo con habilidad, habiendo soltado a Nefertari. La sombra era muy hábil esquivando, pero aún no había atacado, lo que dejaba a Setekh en una tensión constante. La incertidumbre sobre la fuerza del rival solo lo ponía más nervioso.\\
\textit{- Paso de la Luna, numero 6, velocidad sónica}\\
Setekh escuchó aquella frase y, lo siguiente que vio, fue a Nefertari moviéndose a una velocidad superior al sonido, asestándole un golpe a la sombra. El mercenario quedó asombrado al ver cómo Nefertari desprendía un aura de poder pura y tranquila. La sombra voló varios metros fuera de la entrada, lo que provocó que Nefertari cogiera de la mano a Setekh y corrieran rápidamente hacia la entrada de las ruinas.

Una vez dentro, Nefertari se movió más rápido que el sonido por los pasillos de las ruinas, guiando a Setekh hasta la entrada de la guarida de la Luna Creciente. Las sombras quedaban atrás, incapaces de seguir el paso de los dos intrépidos combatientes. Cuando llegaron a la entrada,
Nefertari volvió a sacar la llave con forma de luna, que había usado antes para abrir la puerta ante la que se encontraban, y de forma sútil cuando abrió la cerradura, la guardó de nuevo en su bolsa. Setekh simplemente siguió sorprendido a Nefertari por su actuación en combate y entraron a la guarida, ante ellos se encontraban 10 personas que desprendían bastante poder, uno de ellos se levantó y se acercó a Nefertari para decirle:\\
\textit{- ¿Que ha ocurrido para que hayas tenido que quitar un limitador?}
Nefertari con una cara un poco enfadada, respondió:\\
\textit{- Nos alcanzaron las sombras y empezaron a hacer la materialización de el mercenario que Osahar buscaba}\\
Aquella persona que se levantó se acercó a Setekh a tenderle la mano y decirle:\\
\textit{- Encantado de conocerte, soy Tristan, noble de tierras del este}\\
Setekh dudoso le da la mano y se presenta también, como mercenario de Ankhara.
\textit{- Es un honor conocer al \"Viento desértico del norte\"} - dijo Tristan.\\
A lo que Setekh se sorprendió pero Nefertari, aclaró sus dudas.
\textit{- Todos te conocen por tu nombre del norte, \"Viento desértico del norte\", así fue como Osahar, nos habló de ti, no temas, simplemente Osahar, es capaz de ver más allá de donde ven unos ojos comunes, es normal que sepa tanto sobre ti, debemos ir a hablar con él}\\
Tristan se volvió a sentar donde estaba y Nefertari le hizo una indicación a Setekh para que le acompañase, mientras recorrían la guarida, Nefertari se quejó un poco a Setekh.\\
\textit{- A veces no entiendo como Osahar quiere que haga equipo con esta gente}\\
Con paso tranquilo llegaron a una sala muy cuidada con una mesa y varias estanterías con libros, a la derecha de la sala, se encontraba una persona colocando un par de libros.\\
\textit{- Hemos llegado Osahar, ya te lo he traído, el \"Viento desértico del norte\", ha llegado}\\
Osahar, se giró lentamente y observó a Setekh con detenimiento y asintió, se acercó a la mesa del centro de la sala y dijo:\\
\textit{- ienvenido, Setekh. Hemos estado observando tus habilidades y creemos que tienes un gran potencial para unirte a nuestra causa.}\\
\textit{- ¿Y cuál es exactamente vuestra causa?} - preguntó Setekh\\
\textit{- Nuestro objetivo es proteger Amon-Ra y sus habitantes de las fuerzas oscuras que intentan corromperla. Utilizamos el conocimiento antiguo y la magia para mantener el equilibrio. Pero también buscamos personas como tú, con un potencial único, para ayudarnos en esta misión.} - respondió con convicción Osahar\\
\textit{- Setekh, tu talento y habilidades te han traído hasta aquí. Ahora depende de ti decidir si quieres unirte a nosotros.} - interrumpió Nefertari\\
Setekh miró a Osahar y le respondió a su propuesta:\\
\textit{- Bien, formaré parte de vosotros, no tengo muchas más opciones ahora mismo}\\
Osahar muestra una breve sonrisa de satisfacción antes de gesticular a Nefertari que se retire de la sala. Una vez solos, Osahar se acerca a Setekh y empieza a explicarle más sobre la razón de su elección.\\
\textit{- Setekh, hay algo que necesitas saber. No es solo tu habilidad con las armas lo que nos ha llamado la atención. En ti reside un talento oculto, uno que ni siquiera tú has descubierto completamente.}\\
\textit{- ¿Talento oculto? ¿De qué estás hablando?} - respondió Setekh
Osahar hizo una pausa, contemplando la mejor manera de transmitir la importancia de lo que iba a decir.\\
\textit{- Hace mucho tiempo, cuando las pirámides de Amon-Ra fueron erigidas, no solo fueron construcciones monumentales. Eran centros de poder, lugares donde se concentraba una energía mística. Los faraones de antaño conocían este secreto y, con la ayuda de los dioses, crearon guardianes para proteger este poder.}\\
Osahar hizo un gesto hacia un mural en la pared, que representaba antiguas escenas de faraones y dioses interactuando con fuerzas místicas.\\
\textit{- Estos guardianes, conocidos como los Elegidos, tenían la capacidad de controlar tanto la luz como la oscuridad. Su misión era mantener el equilibrio y proteger las pirámides y sus secretos. Sin embargo, con el tiempo, muchos de estos conocimientos se perdieron, y las sombras del desierto comenzaron a surgir.}\\
\textit{- ¿Qué tienen que ver las sombras del desierto con todo esto?} - preguntó Setekh\\
\textit{- Las sombras del desierto son los restos del poder oscuro que un discípulo del dios de la muerte no pudo controlar. Cuando murió en el desierto oeste, sus sombras quedaron libres, probando a aquellos que se atreven a desafiar su territorio. Estas sombras buscan desequilibrar el balance de poder y corromper a los débiles.}\\
Setekh sintió un escalofrío al recordar las sombras que había enfrentado.
\textit{- Creemos que tú, Setekh, tienes dentro de ti el potencial para ser uno de estos Elegidos. Aún no sabemos que sabes y que no controlar, pero podremos ir viéndolo tranquilamente, además, necesitamos guerreros que puedan enfrentarse a las sombras sin sucumbir a ellas}\\
\textit{- Y cómo sabes que soy yo? ¿Qué pasa si estás equivocado?} - preguntó, intentando asimilar la magnitud de lo que se le estaba diciendo.\\
\textit{- No estamos equivocados. Hemos visto indicios en tus acciones, en tus luchas, y en tu espíritu. Las pruebas a las que te someterás te revelarán tu verdadero poder. Pero debes estar dispuesto a aceptar este camino, con todos sus desafíos y peligros}\\
Setekh, pensativo, sintió el peso de la decisión que estaba tomando.\\
\textit{- Si esto es verdad, entonces tengo mucho que aprender. Estoy dispuesto a intentarlo, a descubrir de qué soy capaz.}\\
\textit{- Excelente. Esta noche, comenzaremos tu entrenamiento. Prepárate, Setekh, porque este es solo el comienzo de tu viaje. La Luna Creciente te dará las herramientas y el conocimiento, pero dependerá de ti cómo los uses}\\
Nefertari regresó a la sala, con un semblante tranquilo pero con la determinación clara en sus ojos.\\
\textit{- Nefertari, prepárate para iniciar el ritual. Setekh, síguela. Te espera una noche larga y llena de revelaciones}\\
Setekh asintió, y mientras seguía a Nefertari, no podía evitar sentir una mezcla de anticipación y nerviosismo. Había algo en Osahar, una calma calculadora que le hacía desconfiar, pero también una promesa de poder y conocimiento que no podía rechazar.
Setekh no venía de una familia adinerada tampoco que fuera capaz de darle otro futuro, más que el que tiene ahora, si bien es cierto que la vida de mercenario le ha ayudado a sobrevivir, aún el joven mercenario no tiene el poder suficiente para poder hacer frente
a un guardia normal y corriente, mientras que Nefertari, es capaz de eso y bastante más, ¿Quién sabe a que limite puede llegar Nefertari?, solamente con una habilidad antes fue capaz de lanzar varios metros por el aire a una sombra de un elevado poder.

Nefertari estaba nerviosa por la noche que le esperaba a Setekh, a pesar de haber llegado recientemente, aún se estaba acomodando en su propia sala,
escogida por Osahar, esta sala, era algo simple, algo así como una habitación de un tamaño medio, con un sitio para dormir, un pequeño escritorio y alguna estantería con libros, pareciera que hubo alguien viviendo ahí antes.
Setekh en cambio no estaba tan nervioso, pero si tenía un peso sobre sus hombros considerable, sabía que en breves la prueba de Osahar iba a comenzar y aún no sabía que debería realizar, Setekh además era consciente de que no es capaz de
ganar en combate a ninguno de los allí presentes con solo sus habilidades de mercenario, por lo que se sentía el más débil de la guarida por el momento. Más pronto que tarde, apareció Osahar, y le dijo a Setekh que estaba todo preparado para probar
sus capacidades, Setekh con algo de nerviosismo asintió y siguió a Osahar, el cual con paso tranquilo le llevo a una sala muy grande, en la cual el espacio era simplemente abrumador, a lo lejos en la otra punta de la sala se divisaba otra persona, antes de divisar quien era, Osahar dijo unas palabras:\\
\textit{- Bienvenido a la sala de entrenamiento de la guarida, en esta sala probaremos que eres digno de ser el elegido que buscamos, para ello, te vas a enfrentar en combate a uno de mis mejores guerreros, Lucian Drakon, caballero del magma}\\
\textit{- ¿Estás demente Osahar?,¿Cómo vas a enfrentar a Setekh contra Lucian?} - exclamó Nefertari\\
\textit{- Nefertari, no dudo de las capacidades de Setekh, es por ello que quiero realizar esta prueba, tu solo, sientate y observa el combate atentamente.} - respondió Osahar de forma contundente.\\
Setekh tenía dudas, claramente visibles para Osahar, quien decidió intervenir antes de que comenzara el combate. Osahar le entregó una espada a Setekh, quien la miró con confusión. Nunca había manejado una espada, y el peso del arma parecía una idea suicida frente a Drakon. Aún así, Osahar insistió.\\
\textit{- Debes ganar a Drakon en un duelo. Puedes usar cualquier técnica que conozcas, y esta espada es muy parecida a la de Drakon, tiene el mismo peso.}\\
Osahar se dirigió al palco, dejando a Setekh solo en la sala. A lo lejos, Lucian Drakon se mantenía firme, con una mirada altiva y arrogante. Setekh no sabía lo que le esperaba, pero podía sentir la presión aumentar con cada paso que daba. Cuando Osahar llegó al palco, Lucian desenfundó su espada, y la hoja se envolvió en llamas danzantes. El calor era palpable, y Setekh sintió un escalofrío recorrer su columna a pesar del calor.\\
\textit{- ¿Listo para morir?} - dijo con desdén, levantando su espada.\\
Setekh intentó levantar la espada para ponerse en guardia, pero antes de que pudiera reaccionar, Drakon estaba frente a él, pasando la hoja llameante cerca de su rostro. Setekh apenas esquivó el primer ataque, pero Drakon no perdió la oportunidad de burlarse.\\
\textit{- Has tenido suerte de que no haya calentado. El próximo ataque acertará sin problemas, prepárate}\\
Setekh se puso en guardia, pero Drakon rompió su defensa con un golpe rápido, enviando la espada de Setekh volando varios metros. Setekh, en un instinto de supervivencia, sacó la daga que Nefertari le había dado. Drakon esbozó una sonrisa de desprecio al ver la pequeña arma.\\
\textit{- ¿Una daga? Qué patético}
Drakon atacó de nuevo, pero Setekh logró parar el golpe con la daga. En ese instante, el fuego que había desaparecido de la hoja de Drakon salió de forma salvaje, empujando a Setekh contra la pared. El joven mercenario quedó sentado en el suelo, inconsciente a primera vista. Drakon alzó su espada, preparándose para el golpe final.

Osahar observaba con atención y se percató de algo, una pequeña esencia de corrupción salió del cuerpo de Setekh.

Drakon inició su carga y asestó el golpe. Nefertari se llevó las manos a la cabeza, dejando de mirar. Pero cuando Drakon terminó su ataque, no había cuerpo.

\textit{- ¿Lo habré carbonizado?}

De repente, alguien le tocó la espalda. Drakon se giró para encontrar a Setekh, pero algo había cambiado. Sus ojos ardían con una fuerza desconocida.

\textit{- Te felicito, has conseguido esquivar un ataque estando muer....} - Antes de terminar la frase, recibió un puñetazo en el estómago que lo envió a la pared.

Drakon, se levantó con daños de la pared y escupiendo algo de sangre, aun así se alzó de nuevo la espada poniéndose en guardia, Setekh había conseguido fuerza de una manera extraña, antes de que Drakon iniciase otro ataque, Setekh abrió la boca y dijo:

\textit{- Paso de la Luna, numero 6, velocidad sónica}

El aire vibró con la energía que emanaba de Setekh mientras se movía a una velocidad asombrosa. Drakon, sorprendido por la repentina transformación de su oponente, apenas tuvo tiempo de reaccionar antes de ser golpeado por una ráfaga de ataques rápidos y precisos.

Desde el palco, Osahar y Nefertari observaban con incredulidad. No esperaban que Setekh pudiera desplegar tal nivel de habilidad y velocidad. Osahar, conocedor de los secretos antiguos y las técnicas avanzadas, comprendió la magnitud del Paso de la Luna número 6. Era una habilidad reservada para aquellos con un dominio excepcional sobre el poder y la velocidad.

\textit{- Impresionante... Su potencial es mayor de lo que pensaba.} - decía para sí mismo Osahar

Nefertari, igualmente impresionada, observaba con orgullo cómo Setekh desafiaba las expectativas y demostraba ser un digno aspirante a entrar en la organización de la Luna Creciente.

Setekh continuó atacando con ferocidad, aprovechando cada apertura y superando a Drakon en velocidad y agilidad. Aunque su cuerpo mostraba signos de fatiga por el uso de este poderoso movimiento, Setekh no se detuvo. Su determinación y habilidad estaban a la vista de todos.

Drakon, aunque impactado por la súbita transformación de Setekh, se mantuvo firme y contraatacó con renovado vigor. La batalla se intensificó, con ambos combatientes desplegando sus mejores habilidades en un duelo que podría decidir mucho más que una simple prueba de fuerza.

La sala resonaba con el choque de acero y la energía mágica, mientras Setekh y Drakon se enfrentaban en un duelo que marcaría el destino de ambos y el curso de la Luna Creciente.

El joven mercenario, impulsado por el Paso de la Luna número 6, había sorprendido a todos con su velocidad y agilidad. Sin embargo, en medio del ardor del combate, una oscura y repentina energía se había apoderado de él. Una sombra de corrupción que alteró su naturaleza y socavó su resistencia. Esta fuerza misteriosa parecía impulsar a Setekh, dándole una ferocidad inesperada en la segunda parte del duelo.

El clímax se acercaba rápidamente. Drakon, sintiendo la necesidad de tomar al novato en serio, ajustó su postura bajo el yelmo y se preparó para el ataque final. El tiempo parecía ralentizarse mientras las llamas danzantes de su espada se estabilizaban. Setekh, por su parte, permanecía impasible, su rostro sin revelar emoción alguna tras levantarse del brutal golpe.

Drakon avanzó con paso firme, moviéndose a una velocidad que sorprendió incluso a Setekh. Estaba a punto de lanzar un golpe decisivo con su espada, que enviaría al joven mercenario rebotando nuevamente. Setekh giró, enfrentando la embestida con su daga, logrando milagrosamente bloquear el violento ataque. Sin embargo, la fuerza del impacto fue tal que lo deslizó hacia atrás, incapaz de detener el impulso.

En medio del deslizamiento, una espada resonó al clavarse en el suelo. Drakon utilizó su espada como punto de apoyo para propulsarse a una velocidad inalcanzable para Setekh. El caballero estaba a punto de asestar un golpe devastador cuando Setekh, aún en movimiento, no pudo reaccionar a tiempo para defenderse.

La patada de Drakon finalizó el combate con un golpe contundente que cambió la trayectoria de Setekh, estrellándolo contra la pared con una fuerza renovada. La nube de polvo que se levantó marcó el final del enfrentamiento, con Setekh inmóvil en el suelo.

Drakon observó con seriedad la escena, sin bajar la guardia mientras la nube de polvo se disipaba. Setekh hizo un intento breve y fallido por levantarse, lo cual alertó a Drakon de que aún podía tener alguna resistencia. Pero finalmente, el joven mercenario cayó rendido, y Drakon fue declarado el vencedor del duelo.

El combate había dejado varias fracturas al mercenario, que inconsciente, no se le veía muy capaz de reponerse, Osahar era consciente de esto y lo llevo a su habitación donde le realizaría cuidados, Drakon, en cambio aunque había ganado, sentía que el novato podía ser un peligro para la organización,
por lo que se lo comunicó a Osahar, advirtiendole que sería capaz de pararlo el mismo en caso de que la situación se saliera de control, esto a Osahar no le hacía mucha gracia y asintió pero no con una cara de convicción. Nefertari habia quedado sorprendida de que Setekh fuera capaz de copiar una técnica que había visto apenas unas horas antes,
aunque todos eran conscientes de que la segunda parte del combate no era Setekh el que estaba peleando, sino la energía corrupta junto con una energía oscura que desconocen de donde proviene, Nefertari acompañó a Osahar para cuidar de Setekh y esperar a que levantará.
