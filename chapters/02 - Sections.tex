\section{El eco de los antiguos}
\subsection{Mercado de Karnak - Desierto de Amon-Ra}
\paragraph{\textit{En las arenas doradas del desierto de Amon-Ra, donde el sol quema sin piedad, y las noches traen promesas de misterio, un joven mercenario llamado Setekh buscaba su próximo bocado
No era un ladrón por elección, sino por necesidad, y en las sombras del mercado de Karnak, su destino se estaba tejiendo en silencio}\\}

El mercado, aquel día se encontraba realmente abarrotado de gente, casi no cabía un alfiler en tan pequeñas calles de la ciudad, y Setekh lo sabía perfectamente. Hoy era uno de esos días en los que,
él podía ganar bastante dinero y lo más importante, un bocado al final del día, era consciente de que hoy, teniendo tales facilidades, los mercaderes, estaría alerta de personas como él. Pero, como dijo el mismo:\\
\textit{- Quien no arriesga, no gana, y hoy debo ir con todo.}\\
Y dentro de esa multitud de gente alrededor de los preciados productos de los mercaderes, se encontraba él, fluyendo entre la gente, y poco a poco haciéndose una pequeña fortuna mientras seguía hacia delante,
a paso lento, pero tampoco muy rápido, simplemente se estaba dejando llevar, entre tanta gente, y estaba llegando al final de la avenida, pocos pasos le separaban de acabar de amasar una fortuna y poder ir libremente,
a poder tener su preciado bocado. Pero una voz detuvo ese paso fluido como si de un baile se tratara, alguien apagó la música en tal \"bella arte\" que tenía el mercenario. \\
\textit{-¡Alto ahí! Suelta todo lo que llevas, te hemos visto desde hace un buen rato, pagarás tus malos actos frente a la justicia.}\\
Setekh, se paró y como si del mismo diablo se tratase, su cara, quedó fría, en aquel ambiente sofocante, donde hasta el mismo aire pesaba del calor que hacía, mientras escuchó aquellas palabras, vislumbró dos guardias dirigiéndose hacia él, y sabía perfectamente
que en su espalda habría otros dos yendo a por él. Se encontraba en una encrucijada de la que era dificil salir, por lo que Setekh, pensó rápido y corrió hacia un callejón, siendo un mercenario y ladrón, debe saberse cada hueco de la ciudad, pero los guardias, igualmente,
conocen cada rincón, y siguieron al mercenario. Sus ropajes de color azabache claro, delataban su presencia, y su mirada entre su turbante, anunciaba sus intenciones, ahora mismo, el juego había cambiado, Setekh debía escapar y esconderse, mientras los guardias seguían tras él,
pero como suelen decir, la codicia, es pesada y Setekh, lo estaba experimentando, debía dejar parte de las monedas de oro que había robado, o de lo contrario, los guardias lo atraparían tarde o temprano, así que, eso hizo, dentro de un ladrillo de la ciudad, escondió una bolsa
con varias monedas de oro, que más tarde recogería, así ahora Setekh podría correr sin preocuparse por ir más lento, pero, el juego del gato y el ratón estaba cerca de su final, y nuestro mercenario, estaba perdiendo la ronda. Setekh estaba siendo más rápido que los guardias, pero
estos, erán muy astutos y lograban bloquear cada callejón al que Setekh iba a ir para escapar, el mismo mercenario no tardó en darse cuenta que poco a poco, estaban cercando su posición y pelear en sus condiciones con un guardia, solo le traería la ruina, solamente quedaba una opción
y era entregarse, pero si lo hacía, sería probable que durmiera en las catacumbas de la ciudad. La situación era realmente complicada y Setekh poco a poco, gracias a su velocidad, iba esquivando los guardias, pero, ¿cuánto tiempo podría estar esquivando mientras el cerco se hacía más pequeño?,
el joven mercenaria no veía salida, cada vez, estaba más rodeado y los pequeños callejones disminuían conforme la guardia iba tras él, estaba cerca de ser capturado, pero, una voz que escuchó, le tendió la mano:\\
\textit{-Parece que vas con prisa, por aquí no vendrán.} \\
Una joven, se dirigió al mercenario para ofrecerle entrar a una pequeña casa entre callejones, a lo que el mercenario, dubitativo no respondió, pero la joven sí:\\
\textit{-No me lo pensaría demasiado, no parece que tengas tiempo, si quieres seguir sin pisar las catacumbas, sígueme}\\
El mercenario, siguió la joven que parecía ser incluso más rápida que él, y usaba un extraño paso, que le fascinó al mercenario, aquellos pasos eran veloces como el viento, y suaves como la brisa de primavera, el mercenario podía mantenerse a su ritmo y pronto esquivando la guardia, llegaron
al lugar donde la joven quería llegar. Una pequeña casa austera, y tranquila, unas calles más abajo de donde se encontró con el mercenario, una vez dentro, la joven se quitó su ropaje de tela que le cubría el rostro y parte del que le cubría el pelo, tras esas prendas de seda, se escondía un largo cabello plateado,
que a la luz del sol brillaba con destellos, junto con unos ojos color verde esmeralda, tan claros como el agua de un río, la joven, mientras dejaba ver su rostro advirtió al mercenario:\\
\textit{-He visto que tienes las manos muy ligeras, caballero, pero dejame decirte que tengo algo que has dejado por ahí tirado, francamente, se te da mal esconder cosas}\\
Tras esto, la joven dejó caer la bolsa de monedas robadas del mercenario, sobre el suelo, a lo que el mercenario quedó sorprendido:\\
\textit{-¿En qué momento has sido capaz de recogerlas?}\\
La joven, con una sonrisa confiada y un brillo pícaro en sus ojos, respondió:\\
\textit{-Vaya, menudos modales. Lo primero que te oigo decir y ni siquiera te presentas.} \\
El mercenario, desconocía la identidad de la mujer, pero aun así agradeció su gesto:\\
\textit{-Tienes razón, además estoy en deuda contigo, soy Setekh, mercenario del desierto norte.}\\
La mujer, sonrió y se dió la vuelta para rebuscar algunas cosas mientras hablaba con el mercenario:\\
\textit{-No hace falta que seas tan correcto, solamente bromeaba contigo, te he estado observando un tiempo en el mercado y preveía que tendrías problemas, además, hay alguien que tiene interés en ti, por lo que,
tarde o temprano debería conocerte en persona, \"Viento desértico del norte\"}\\
Los ojos del mercenario, en un momento se volvieron platos, su sorpresa fue mayúscula pero mantuvo la compostura:\\
\textit{-¿Cómo conoces ese nombre?, ¿Quién eres?, ¿Quién me busca?}\\
La mujer volvió a mirarle y le dijo:\\
\textit{-Para ser tan misterioso, preguntas bastante, pero, ¿volvemos al interrogatorio?, eres curioso}\\
Setekh, no parecía entender muy bien que es lo que deseaba esta chica, y presentía algo malo en todo esto, su mente podía estar jugandosela, pero quizá no era una mala idea confiar en la desconocida mujer:\\
\textit{-Puedes prescindir de tus ropajes, no hace falta que ocultes más tu rostro, aquí dentro no hay peligro, soy Nefertari, pertenezco a una organización que busca personas con talentos ocultos. Nuestro líder cree que tienes un potencial especial, uno que ni siquiera tú conoces todavía. Mi tarea era encontrarte y traerte con nosotros.}\\
Setekh frunció el ceño, evaluando las palabras de Nefertari. La desconfianza se mezclaba con la curiosidad en su mirada:
\textit{-¿Y por qué debería confiar en ti? Podría ser una trampa}\\
\textit{-No te pido que confíes ciegamente. Te ofrezco una oportunidad para descubrir quién eres realmente y para qué estás destinado. Si aceptas, podrías encontrar respuestas y poder que nunca imaginaste. Si no, eres libre de irte con tus monedas robadas.}\\
Setekh se quedó en silencio por un momento, pensando en su vida hasta ahora y en las posibilidades que Nefertari ofrecía. Había dos opciones sobre la mesa y el joven mercenario debía decidir, quizás una decisión que encaminaría su vida a algo más grande. La primera opción parecía bastante intersante pero Setekh aun desconfía de una desconocida, aunque la segunda opción no era mejor, por lo que Setekh suspiró y con mucha calma dijo sus condiciones:\\
\textit{-De acuerdo. Te seguiré. Pero si esto es una trampa, te aseguro que no será fácil atraparme}\\
\textit{-No esperaba menos de ti, Setekh. Ven, tenemos mucho que discutir y preparar.}\\

\clearpage 
% ------------------------------------------------------
Un tiempo más tarde...
Nefertari y Setekh, se conocieron algo más y ella, le otorgó más detalles acerca de la organización de la que formaba parte, conforme pasaba el tiempo, Setekh estaba más convencido de entrar a la organización y Nefertari, dejaba ver muestras de felicidad mientras Setekh se mostraba más convencido. Ambos esperaron a la noche y Nefertari, le dijo a Setekh que deberían partir por el paso subterráneo de la ciudad, a lo que Setekh replicó, alegando que siempre hay guardias, lo que no sabía Setekh,
es que Nefertari se refería a otro paso subterráneo, uno, más escondido:
\textit{-Por aquí, sígueme}\\
Setekh seguía a Nefertari, sin ninguna objección y estaban yendo en dirección oeste de la ciudad, el paso subterráneo, se encontraba en la parte norte, pero Setekh se limitó a seguir a Nefertari. Pronto llegaron a un pequeño monumento en honor a Khonsu, el dios egipcio de la luna y la noche. El monumento era una pequeña fuente con una piedra que por las noches, emite un brillo especial que baña el agua de la misma en un color verde bosque por su reacción con la luz de la luna. Nefertari dejó caer en el suelo un puñado de arena que sacó de una bolsa, y puso su mano boca abajo encima del lugar donde cayó la arena, se arrodilló y pronto se iluminó poco a poco un circulo alrededor de ella y Setekh, era un circulo color cobre con algunos jerogrificos
escritos, poco a poco se fue levantando la arena que Nefertari había soltado, y fue rodeando a los dos, Nefertari se levantó y dijo:\\
\textit{-Ahora se abrirá una puerta, deberemos pasar rápido por ella, simplemente no dudes, necesito que pases con convicción, de lo contrario, no llegaremos al paso}\\
Setekh asintió, y tan pronto como lo hizo, el suelo de sus pies, empezó a ceder y bajar. Setekh se sorprendió pero Nefertari le pidió que no se moviera, estaban entrando al paso subterráneo secreto del oeste. El suelo siguió bajando hasta que llego la plataforma del paso, donde se encontraban dos antorchas iluminando la entrada, Nefertari, salió de la plataforma y Setekh la siguió, ahí se encontraba otra puerta más, ante la cual, Nefertari y le dijo:\\
\textit{- ¿Sabes?, sé que es repentino y quizás no lo esperas, pero, a partir de aquí, ten cuidado, iremos hasta la \"Luna Creciente\", aunque te voy a acompañar, cruzar este paso, no es solo un camino que debemos recorrer, es una decisión de no volver atrás, por eso quiero, que pase lo que pase, te mantengas fuerte}\\
Setekh solamente asintió y cruzó la puerta con ella, dentro del paso, la presión era inmensa, apenas se podía respirar debido a una presión de poder que recorría todo el paso, Nefertari se estremeció un poco y dijo:\\
\textit{- Lo que sientes es presión de poder pura, viene del centro de la gran pirámide, por aquí pasa una raíz de poder de ella, de lo único de lo que nos debemos preocupar es de la acumulación de armadura de corrupción que hay por este lugar, simplemente mantente alejado, y no correrás peligro}\\
Setekh y Nefertari, siguieron el camino del paso, sin mucha prisa, a un paso calmado, ya que la presión de poder apenas permitía ir más rápido, aun así, llegaron al fin del paso, en el que esperaba, de nuevo una puerta, pero esta, era distinta, requería usar una llave.
Nefertari sacó de su bolsa, una llave con la forma de una luna creciente, que introdujo en la cerradura de la puerta lentamente y abrió sin mucho esfuerzo, ambos pasaron la puerta y Nefertari, se aseguró de cerrar con llave aquella puerta, la presión de poder seguía ahí pero no era tan grande como en el paso.\\
\textit{- Siempre es complicado pasar por aquí por culpa de la presión, a veces la gente se ha desmayado ya que no son capaces de aguantar la presión}\\

